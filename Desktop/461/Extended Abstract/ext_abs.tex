\documentclass[12pt]{article}
\usepackage[affil-it]{authblk}
\usepackage{graphicx}
\usepackage{amsmath}
\graphicspath{ {} }
\setcounter{secnumdepth}{0}
\usepackage{cite}


\begin{document}

\title{Level 1 probabilistic risk analysis of a molten salt reactor}

\author{Anshuman Chaube}%
%   \thanks{Email:\texttt{anshumanchaube95@gmail.com}; }}
\affil{NPRE 461 Project: Extended Abstract}

\date{}

\maketitle

\begin{abstract}
While most molten salt reactor designs are inherently proof against many design based accidents that most light water and presurrized water reactors are susceptible to, systematic studies are required to characterize the safety and reliability of these designs. The focus of this project has the estimation of probability of overheating of the MSR core, resulting in the melting of core components and release of molten salt and gaseous fission products. The major MSR components contributing to such an eventuality are described, and the method of estimation of probability is outlined.
\end{abstract}

\section{Introduction}

Molten Salt Reactors are being studied by the Generation IV forum as one of the safest and most proliferation resistant designs of all the other candidate reactors. The molten salt fuel and moderator creates many passibe safety features and makes the reactor designs immune to many design based accidents that are likely in conventional pressurized water and light water reactors \cite{elsheikh_safety_2013}. Such accidents include:
\begin{itemize}

\item \textbf{Core Meltdown} : Since the reactor core comprises of molten salt, there is no risk of meltdown. As the salt heats up, reaction rate decreases due to the large negative reactivity coefficient of fuel-salt. If it overheats, the salt can be drained into secure tanks beneath the reactor where it may cool off as the reactions in it die down.

\item \textbf{Control Rod Ejection and Reactivity Initiated Accident}: The occurence of this accident due to mechanical or operator error creates excess reactivity in conventional reactors. However, as mentioned previously, the negative coefficient of reactivity precludes any major mishaps in MSRs due to such an accident. 

\item \textbf{Criticality Accident} : Excess reactivity can be easily controlled due to the negative temperature coefficient of the salt. The reactivity can be controlled without control rods. Online fuel processing keeps a tight check on the amount of fuel in circulation. The core design and geometry is also optmized such that there is no external structure that can be compromized.

\end{itemize}

\section{Level 1 PRA: Core overheating scenario}

\subsection{Possible contributors}

However, due to the unique design and operating requirements of the MSR, certain components are vulnerable to failure and can result in overheating of the core. It is possible for the core components to melt in such circumstances, allowing molten salts and gaseous fission products to escape the reactor. This can cause radiation poisoning and boron poisoning in the immediate vicinity of the reactor and release of radioactive material to the public at large.

The major contributors to such a scenario are:

\begin{itemize}

\item \textbf{Pump failure} : The failure of pumps in either the primary or the secondary loop can result in loss of circulation of salts. This effectively stops heat transfer from the core and can very quickly result in core damage \cite{elsheikh_safety_2013} \cite{yoshioka_guidelines_2012} .

\item \textbf{Sparging system failure} : While most fission products remain dissolved in the fuel salts, the MSR design requires the injection of He into the salts to remove gaseuos fission products such as Xe and Kr to prevent neutron poisoning. The shutdown of the sparging system can cause the core to evolve an excess of neutron poisons. However, excessive insertion of He can cause positive reactivity to be inserted which further heats the core \cite{yoshioka_guidelines_2012} \cite{moir_recommendations_2008}.

\item \textbf{Salt freeze in secondary loop}: The molten salt is kept above its freezing point by electrically heating the pipes in the secondary loop, as it was done in the molten salt reactor experiment at Oak Ridge National Laboratory. The backup system that must maintain this temperature during transients can fail, causing the salt temperature to fall below its melting point in certain sections of the secondary loop. The solidified salt can accrue and eventually create a blockage, causing loss of heat transfer similar to pump failure \cite{haubenreich_experience_1970}.

\item \textbf{Excess fuel in core} : The fuel reprocessing system is responsible for maintaining the concentration of fuel dissolved in the molten salt and for removing dissolved fission products. Insertion of excess fuel in the salt can cause excess reactivity to be inserted, overheating the core \cite{haubenreich_experience_1970}.

\item \textbf{Drainage system failure}: For a given core that has overheated, the final fail safe is the freeze plug and drainage tank system. During normal operation, a section of the pipes above the drainage tanks is kept blocked by cooling it and solidifying a portion of the molten salt. As the core overheats, this freeze plug should melt and open the drainage pipes, allowing the overheated molten salt to safely pass into the drainage tanks where it can cool. This final line of defence can fail if (i) the pipes have manufacturing defects such as non-uniform cross section that is smaller than specified, preventing drainaige of the core in a reasonable amount of time (ii) the cooling system for the drainage tanks fails, allowing nuclear reactions to go on in the salt and failing to cool it before it leaks\cite{elsheikh_safety_2013}.


\end{itemize}

With these vulnerabilities in mind, it is possible for the reactor to reach a state of catastrophic failure in many ways. One possible consequence, overheating of the core followed by molten salt leak, is analyzed herein.

\subsection{Analysis method}

The major modes of overheating the core are excess reactivity insertion and loss of cooling, and the simultaneous failure of the drainage system failsafe. Their relationship can be modelled using a fault tree.

Excess reactivity can be inserted by either insertion of excess He from the sparging system or by addition of excess fuel in the molten salt through the reprocessing system. Both are possible due to pumps malfunctioning, the data of which can be obtained from IAEA PRA databases \cite{iaea_component_1988}, or due to human error, which can be estimated using THERP. Failure of primary loop's circulation pumps can similarly create excess reactivity, which can again be modelled using the PRA databases.

The loss of cooling can occur due to freezing of salts after failure of the heating generators, due to pipe breakage or due to secondary pump failure, each of which can be quantified using the aformentioned databases.

Finally, drainage system failure can be quantified as the probability of manufacturing defects in the pipe that cause clogging, or failure of the cooling system in the drainage tanks.


\bibliography{ext_ab_ref}
\bibliographystyle{abbrv}





\end{document}

