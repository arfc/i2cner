\section{Introduction} \label{Introduction}
In order to mitigate climate change and to improve environmental outcomes, many nations are actively seeking to reduce carbon emissions, and have formalised this goal through the Paris Agreements \cite{united_nations_framework_convention_on_climate_change_unfccc_submission_2015}. The largest contribution to global \gls{GHG} emissions, some 73\%, comes largely from energy consumption in the transportation, electricity and heat, buildings, manufacturing, and construction sectors \cite{ge_4_2020}. Developed nations rich in natural resources can reduce \gls{GHG} emissions by switching to natural gas or implementing \gls{CCS} at fossil fuel plants. However, these options are often economically infeasible for developing nations, which will lead to increased emissions through greater coal use \cite{international_energy_agency_latest_2019}. Japan, a developed nation without fossil fuel resources, is likely to follow a different path to reduce emissions, evidenced by the restart of nuclear reactors and a shift towards large-scale renewable energy deployment \cite{international_energy_agency_latest_2019}.

Although influenced by the Paris Agreements, Japanese energy policy is governed by the Basic Energy Plan \cite{meti_japans_2018}, which outlines national policy towards a new energy system for the years 2030 and 2050 cognizant of limited indigenous resources, the impact of the Fukushima incident, and external pressures on energy supplies \cite{meti_annual_2018}. The plan reaffirms Japanese benchmarks for evaluating the energy system, first and foremost, within the context of energy security, followed by economic efficiency, safety, and the environment (summarised as `3E+S'; ibid). Although the Japanese 3E+S goals and the Paris Agreement targets have some parallels, the plan does not detail how the 2050 emission reduction target of 80\% is to be met. Matsuo et. al have suggested that electrification of a number of sectors will be required to achieve the ambitious 2050 target, underpinned by low-carbon technologies \cite{matsuo_quantitative_2018}. For the power sector to achieve such a target, near-zero emissions are required, and early action utilising existing technologies is preferable to delayed action utilising future technologies \cite{ashina_roadmap_2012}. The strategies currently under consideration include reinvigorating nuclear power, deploying \gls{CCS} to fossil fuel power plants, and ushering in a hydrogen economy based on renewable energy-based electrolysis as well as hydrogen imports from abroad \cite{ashina_roadmap_2012, matsuo_quantitative_2018, meti_basic_2017}. 

This research aims to investigate the likely suite of electricity generation and storage technologies, as well as their feasibility in meeting Japan's carbon reduction goals, while being cognizant of energy policy, resource limitations, demand growth, emerging technologies, and economic constraints using the \gls{TIMES} framework. Our dynamic simulations of transition scenarios, which focus on minimising the cost of the transition while satisfying CO$_2$ emission constraints, suggest potential economically feasible decarbonisation pathways that meet the increasing near-term electricity demand. Additionally, we assess the significance of key economic parameters of emerging technologies through sensitivity analysis, in order to highlight the most impactful parameters and hence guide research and development efforts focused on these technologies.
