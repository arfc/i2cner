%%%%%%%%%%%%%%%%%%%%%%%%%%%%%%%%%%%%%%%%%
% Jacobs Landscape Poster
% LaTeX Template
% Version 1.1 (14/06/14)
%
% Created by:
% Computational Physics and Biophysics Group, Jacobs University
% https://teamwork.jacobs-university.de:8443/confluence/display/CoPandBiG/LaTeX+Poster
% 
% Further modified by:
% Nathaniel Johnston (nathaniel@njohnston.ca)
%
% This template has been downloaded from:
% http://www.LaTeXTemplates.com
%
% License:
% CC BY-NC-SA 3.0 (http://creativecommons.org/licenses/by-nc-sa/3.0/)
%
%%%%%%%%%%%%%%%%%%%%%%%%%%%%%%%%%%%%%%%%%

%----------------------------------------------------------------------------------------
%	PACKAGES AND OTHER DOCUMENT CONFIGURATIONS
%----------------------------------------------------------------------------------------

\documentclass[final]{beamer}

\usepackage[scale=1.0]{beamerposter} % Use the beamerposter package for laying out the poster
\usetheme{confposter} % Use the confposter theme supplied with this template

\setbeamercolor{block title}{fg=dblue!80,bg=white} % Colors of the block titles
\setbeamercolor{block body}{fg=black,bg=white} % Colors of the body of blocks
\setbeamercolor{block alerted title}{fg=white,bg=dblue!70} % Colors of the highlighted block titles
\setbeamercolor{block alerted body}{fg=black,bg=dblue!10} % Colors of the body of highlighted blocks
% Many more colors are available for use in beamerthemeconfposter.sty

%-----------------------------------------------------------
% Define the column widths and overall poster size
% To set effective sepwid, onecolwid and twocolwid values, first choose how many columns you want and how much separation you want between columns
% In this template, the separation width chosen is 0.024 of the paper width and a 4-column layout
% onecolwid should therefore be (1-(# of columns+1)*sepwid)/# of columns e.g. (1-(4+1)*0.024)/4 = 0.22
% onecolwid should therefore be (1-(# of columns+1)*sepwid)/# of columns e.g. 
% (1-(3+1)*0.025)/3 = 0.3
% Set twocolwid to be (2*onecolwid)+sepwid = 0.464
% Set threecolwid to be (3*onecolwid)+2*sepwid = 0.708
\newcommand{\Cyclus}{\textsc{Cyclus}\xspace}%

\newlength{\sepwid}
\newlength{\onecolwid}
\newlength{\twocolwid}
\newlength{\threecolwid}
\setlength{\paperwidth}{36in} % A0 width: 46.8in
\setlength{\paperheight}{48in} % A0 height: 33.1in
\setlength{\textwidth}{34in} % A0 width: 46.8in
\setlength{\textheight}{46in} % A0 height: 33.1in
\setlength{\sepwid}{0.025\paperwidth} % Separation width (white space) between columns
\setlength{\onecolwid}{0.3\paperwidth} % Width of one column
\setlength{\twocolwid}{0.625\paperwidth} % Width of two columns
\setlength{\threecolwid}{0.95\paperwidth} % Width of three columns
\setlength{\topmargin}{-0.5in} % Reduce the top margin size
%-----------------------------------------------------------

\usepackage{graphicx}  % Required for including images

\usepackage{tabularx}
\newcolumntype{b}{X}
\newcolumntype{s}{>{\hsize=.5\hsize}X}
\newcolumntype{m}{>{\hsize=.75\hsize}X}
\newcolumntype{z}{>{\hsize=.65\hsize}X}

\usepackage{booktabs} % Top and bottom rules for tables
\usepackage{xspace}

\usepackage{tikz}
\usetikzlibrary{positioning, arrows, decorations, shapes, calc }
% Define block styles
\tikzstyle{decision} = [diamond, draw, fill=blue!20, 
text width=4.5em, text badly centered, node distance=3cm, inner sep=0pt]


\tikzstyle{block} = [rectangle, draw, text centered, fill=blue!20]
\tikzstyle{line} = [draw, -latex']
\tikzstyle{cloud} = [draw, ellipse,fill=red!20, node distance=6em,
minimum height=2em]



\usetikzlibrary{shapes.multipart}
\usetikzlibrary{positioning}


\setbeamertemplate{bibliography item}[text]

%----------------------------------------------------------------------------------------
%	TITLE SECTION 
%----------------------------------------------------------------------------------------

\title{\textbf{Dynamic Transition Analysis with TIMES:}\\ I\textsuperscript{2}CNER Initiative on Challenges in Energy Assessment and Energy Transtitions} % Poster title

\author{Kathryn D. Huff, James Stubbins}
\institute{University of Illinios at Urbana-Champaign, Department of Nuclear, Plasma, and Radiological Engineering, Urbana, IL 61801}
%----------------------------------------------------------------------------------------

\begin{document}

\addtobeamertemplate{block end}{}{\vspace*{2ex}} % White space under blocks
\addtobeamertemplate{block alerted end}{}{\vspace*{2ex}} % White space under highlighted (alert) blocks

\setlength{\belowcaptionskip}{2ex} % White space under figures
\setlength\belowdisplayshortskip{2ex} % White space under equations

\begin{frame}[t] % The whole poster is enclosed in one beamer frame

\begin{columns}[t,totalwidth=\threecolwid] % The whole poster consists of three major columns, the second of which is split into two columns twice - the [t] option aligns each column's content to the top

\begin{column}{\sepwid}\end{column} % Empty spacer column

%=======================================================
% FIRST COLUMN BEGINS
%=========================================================


\begin{column}{\onecolwid} % The first column

%----------------------------------------------------------------------------------------
%	OBJECTIVES
%----------------------------------------------------------------------------------------

\begin{alertblock}{Objectives}

Evaluate potential impact of novel energy technologies within Japan's energy system. Specifically:
\begin{itemize}
	\item Help guide practical near term (2010-2050) transition strategies to minimize carbon emissions within realistic constraints.
	\item Identify high impact technologies.
	\item Assess role of technology readiness.
	\item Predict impediments to strategically optimal technology deployment.
	\item Identify ideal timelines for energy system deployment, infrastructure development, high impact R\&D investment.
	
\end{itemize}

\end{alertblock}

\begin{block}{Collaborators}

I\textsuperscript{2}CNER collaborators will include primarily the members of the Energy Analysis Division (EAD), including those members at other institutes, universities or industries with connections to the EAD.   
 
\end{block}

%----------------------------------------------------------------------------------------
%	METHODOLOGY
%----------------------------------------------------------------------------------------

\begin{block}{Methodology}
Multiple studies have been conducted to compare the impact of innovative energy technologies in different regions of the world using \textit{static} scenario analyses \cite{chyong_chi_dynamics_2009} \cite{feng_system_2013} \cite{kikuchi_simulation-based_2017} \cite{li_energy_2010} \cite{pambudi_impact_2017} \cite{pambudi_future_2016}. We will simulate \textit{dynamic transition scenarios} \cite{jebaraj_review_2006} \cite{pfenninger_energy_2014} , with realistic constraints and technology readiness of energy generation technologies (in terms of generation, transmission \& storage), aimed at minimizing carbon emissions. We will further extend previous work by combining multiple technologies in a single heterogeneous system.\\

The TIMES (The Integrated MARKAL-EFOM System) model generator \cite{loulou_documentation_2005} \cite{seebregts_energy/environmental_2002} optimizes energy systems of a model using linear and mixed-linear algorithms while implementing user-defined objective functions (such as minimizing carbon emissions or costs) within user defined constraints such as energy generation demand. It will be used to simulate near-term energy transitions while focusing on reduction of carbon emissions. 

\begin{figure}
	\centering
	\scalebox{0.8}{
		\begin{tikzpicture}[    >=stealth,
		node distance=8cm,
		on grid,
		align=center,
		auto]
		% Place nodes
		\node [block] (times) {\textbf{TIMES Model Generator}};
		\node [cloud, above left=13cm and 12cm of times] (mod) {\texttt{MODEL} \\ heterogeneous \\ mutli-technology \\ model of JPN};
				\node [cloud, above=9cm of mod] (dat) { Data \\ collection};
		\node [cloud, above=9cm of times] (of) {\texttt{OBJECTIVE}\\\texttt{FUNCTION} \\ e.g.: Minimize \\ carbon emissions \\ from all \\ deployed sources};
		\node [cloud, above right=13cm and 10cm of times] (const) {\texttt{CONSTRAINTS} \\ e.g.: Deployed \\
		sources must\\ meet energy \\ demand};
		\node [cloud, below=of times] (res) {Simulation \\ Results};
		\node [cloud, below=of res] (eres) {Projections,\\ Conclusions, \\ Insights} ;
		
		\draw[->, ultra thick] let \p1=($(times)-(mod)$) in (mod) -- +(0,\y1)-- +(times);
		\draw[->, ultra thick] (of) -- (times);
		%\draw[->, ultra thick] let \p3=($(times)-(const)+(2,3)$),\p4=($(times)+(9,9)$),\p5=($(const)-(3,3)$) in (\x3,\y4) -- +(0,\y2)-- +(times);
		\draw[->, ultra thick] let \p2=($(times)-(const)$) in (const)-- +(0,\y2) -- + (times);
		\draw[->, ultra thick] (times) -- (res);
		\draw[->, ultra thick] (res) -- (eres);
		\draw[->, ultra thick] (dat) -- (mod);
 \end{tikzpicture}
		
	}
	\caption{Basic methodology for dynamic simulation of Japan's energy system.}
\end{figure}

\end{block}

%----------------------------------------------------------------------------------------
%	Anticipated Results
%----------------------------------------------------------------------------------------

\begin{block}{Anticipated Results}

 \begin{itemize}
   \item Analysis results can be filtered by sector (commercial, industrial, residential, building etc) or by region.
   
   \item Many metrics are automatically postprocessed- such as energy intensity, thermal energy efficiency, transmission capacity. 
   
   \item Technology deployment transitions driven by constrained optimization will have valuable strategic value.    
 
 
 \end{itemize}  
 
 	
\end{block}



%----------------------------------------------------------------------------------------

\end{column} % End of the first column

\begin{column}{\sepwid}\end{column} % Empty spacer column


%===========================================================================================
% SECOND COLUMN BEGINS
%=============================================================================================

%----------------------------------------------------------------------------------------

\begin{column}{\onecolwid} % The second column



%----------------------------------------------------------------------------------------
%	RESULTS
%----------------------------------------------------------------------------------------


        \begin{alertblock}{Impact}
        \large Results will:
	\begin{itemize}
		\item {\large Assist in creation of realistic decarbonization roadmaps that maximize efficiency.}
        \item {\large Identify potential bottlenecks during transitions.}
		\item {\large Help Japan's policymakers create timelines for R\&D investment and for appropriate infrastructure development.}
		\item {\large Quantify system sensitivity to technology readiness.}
	\end{itemize}
        \end{alertblock}
   

%--------------------------------------------------------
% CHALLENGES
%---------------------------------------------------------
\begin{block}{Challenges}

 \begin{itemize}
   \item Reliable data for each technology's deployment and operation is required, such as:
    \begin{itemize}
    
     \item \normalsize Technology readiness
     
     \item Carbon intensity
     
     \item Capacity and availability factors
     
     \item Fuel costs and demands
     
     \item Thermal/electric generation capacity
     
     \item Storage capacity
     
     \item Thermal efficiency
     
     \item Construction time
     
     \item Construction costs
     
     \item Operation and maintenance costs\\
    
    
    \end{itemize}
    
   
   \item Mutiple constraints cause complications.
   
   \item Variation of goals or the objective function can significantly change each simulation. e.g.: cost could be objective function whereas carbon reduction up to a certain percentage (50-70\%) could be the constraint, flipping the script.
   
   \item Calculation of constituent parameters, such as carbon emissions per component and quantification of deployment, will be complex.
   
   \item Intermediate objective functions for a multi-objective formulation could additionally include intermediate-resolution-goals such as Efficiency Increase (EI) and Lower Carbon Intensity (LCI). These intermediate objectives will increase the complexity of the calculation.
 
 
 \end{itemize}  
 
 	
\end{block}


%----------------------------------------------------------------------------------------
%	CONCLUSION
%----------------------------------------------------------------------------------------

\begin{block}{Summary}
\begin{itemize}

\item Dynamic simulation of Japan's energy system in TIMES model generator using a heterogeneous model and realistic constraints will help develop near-term decarbonization strategies.

\item Policymakers will benefit from identification of high impact technologies, and creation of R\&D investment and infrastructure development timelines.

\item Simulations will quantify system sensitivity to technology readiness, and also account for secondary scenarios where decarbonization is not the main priority.

\end{itemize}

\end{block}

%----------------------------------------------------------------------------------------
%	ACKNOWLEDGEMENTS
%----------------------------------------------------------------------------------------

\setbeamercolor{block title}{fg=norange,bg=white} % Change the block title color

\begin{block}{Acknowledgements}

This research is being performed using funding received
from the I\textsuperscript{2}CNER Programs under award number \textbf{<insert details here>}.

\vspace{10mm}
\begin{center}
\begin{tabular}{cccc}
\includegraphics[scale=0.5]{arfc_logo.png} & \includegraphics[scale=0.5]{wpi_logo.png} & \includegraphics[scale=0.1]{ku_logo.png} &\includegraphics[scale=0.7]{i2cner_logo.png}
%\includegraphics[width=\linewidth, height=0.1\textheight]{i2cner_logo.png}
\end{tabular}
\end{center}


\end{block}

%----------------------------------------------------------------------------------------
%	CONTACT INFORMATION
%----------------------------------------------------------------------------------------

\setbeamercolor{block alerted title}{fg=black,bg=norange} % Change the alert block title colors
\setbeamercolor{block alerted body}{fg=black,bg=white} % Change the alert block body colors



\begin{alertblock}{Contact Information}
\setbeamercolor{block title}{fg=norange,bg=white} % Change the block title color
\begin{itemize}
	
	\item Web: \href{arfc.github.io}{arfc.github.io}
	\item Email: \href{mailto:kdhuff@illinois.edu}{kdhuff@illinois.edu}
\end{itemize}

\end{alertblock}

%----------------------------------------------------------------------------------------

\end{column} % End of column 2

\begin{column}{\sepwid}\end{column} % Empty spacer column

\begin{column}{\onecolwid} % The third column


\begin{block}{References}

        {\footnotesize\bibliographystyle{abbrv} 
        \bibliography{poster}}
\end{block}


%----------------------------------------------------------------------------------------



\end{column} % End of the third column

\end{columns} % End of all the columns in the poster

\end{frame} % End of the enclosing frame

\end{document}
\begin{column}{\sepwid}\end{column} % Empty spacer column
