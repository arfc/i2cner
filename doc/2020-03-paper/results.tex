\section{Results} \label{Results-and-discussion}
\subsection{Transition Scenarios}
Notes for Ansh: electricity diverted to storage could be reduced to text with percentages i.e. "total amount of electricity diverted to storage=, electricity for storage came from nuclear (50\%), solar(30\%),wind(20\%) over 2017-2100". Same for H2 (total GWh generated, amount).

Talk about description of plots, common features, oscillations, storage,

Scenario 1 results (fig. \ref{scen1}) demonstrate our model's attempt at decarbonization using existing technologies without deploying new nuclear power. In all scenarios, coal and oil must be retired by 2030. Natural gas sees rapid growth in the near-term, and complete retirement by 2055. Once deep emission cuts have been achieved, new natural gas is deployed again from 2071 onwards. All existing nuclear power plants must restarted by 2022 at full operating capacity. With renewable energy as the only option for decarbonization, significant investments in solar, onshore wind, offshore wind (both fixed-bottom and floating), and lithium ion storage are necessary. The presence of a large share of renewables results in significant overgeneration of electricity during some years. Due to reduced flexibility caused by the retirement of natural gas and an increase in the share of renewables, nuclear power plants must be able to load follow, which is simulated in our model. The plot displaying the electricity supplied to the end-user shows a high degree of variability as excess electricity from renewables, hydropower, and nuclear power is often diverted to lithium-ion storage as appropriate, instead of being directly supplied to the \gls{TIMES} process. Despite the deployment of a large amount of renewables, the model fails to achieve the 2030 and 2050 emission reduction goals by <AMT1> and <AMT2> respectively. Furthermore, emissions continue to rise after 2050 primarily due to life-cycle emissions of lithium-ion batteries. The amount of electricity diverted to storage technologies over the entire simulation time frame is 46,342 TWh, primarily from solar (36\%), onshore wind (22\%), fixed-bottom offshore wind (22\%),and floating offshore wind (12\%). The total cost of this transition is \textbf{insert number here.}


With the option of deploying new nuclear reactors available in Scenario 2 (fig. \ref{scen2}) , the model chooses to rapidly deploy nuclear power plants at the maximum allowed growth rate despite the high investment cost of nuclear, due to its low life-cycle emissions. Due to the reduction in emissions caused by nuclear power, natural gas power plants can continue to operate until 2100. The amount of renewables required for decarbonization is reduced dramatically. This, combined with the load following capabilities of natural gas plants and nuclear reactors, drastically reduces the amount of lithium ion storage deployed. The model is able to achieve its 2030 and 2050 decarbonization goals, and automatically decarbonizes more than the required limit by 2100. The amount of electricity diverted to storage technologies is 12,220 TWh, primarily from nuclear (46\%), solar (41\%), and onshore wind (8\%). The total transition cost is \textbf{insert number here.}

Using emerging technologies without new nuclear power (fig. \ref{scen3}), the model needs to restart Japan's existing nuclear power plants at full capacity by 2030. The deep emission cuts achieved through renewables, hydrogen and \gls{CCS} leave room for emissions from \gls{lng}, hence natural gas plants continue to operate until 2100. Expansion of solar and onshore wind, along with a modest deployment of lithium ion batteries, helps the model meet 2030 emission goals. After that, the model relies primarily on renewables and hydrogen to curb emissions while supplying power. Investment in hydrogen allows effective utilization of renewables and precludes investment in offshore floating wind power. \gls{lng}-based \gls{CCS} plays a modest role between <YEAR1-YEAR2> as an intermediate technology before the model can complete the transition to utility scale hydrogen. The model deploys <X> GW of \gls{CCS} technology that results in <Y> amount of CO$_2$ captured. As the existing photovoltaic technology approaches the end of its lifetime and emerging solar technologies become cheaper and more efficient, they rapidly replace current solar power, benefiting from the existing solar manufacture and supply chains. The model is able to hold the emissions constant until 2100, but addition decarbonization as in scenario 2 is not achieved. The amount of electricity diverted to storage technologies is 43,879 TWh, primarily from solar and emerging solar technologies (48\%), fixed bottom offshore wind (28\%),and onshore wind (20\%). The amount of hydrogen generated is 35,478 TWh, initially from alkaline electrolysis(1\%), but rapidly transitioning to PEM electrolysis(99\%). The cost of this transition is <insert number here>.

Using emerging technologies with new nuclear power deployment results in rapid decarbonization that continues until 2100. The model achieves an additional <X> Mt of CO$_2$ reduction by 2050. The deployment of around 50 MW nuclear obviates the need to invest in offshore wind, lithium ion storage, and \gls{CCS}. Hydrogen plays a significant role in decarbonization, but it is deployed from <YEAR> instead. The amount of electricity diverted to storage technologies is 29,733 TWh, primarily from solar and emerging solar technologies(62\%), onshore wind (20\%), and nuclear (13\%). The amount of hydrogen generated is 24,264 TWh, produced entirely from PEM electrolysis. The cost of this transition is <insert number here>.

\begin{figure}[htb] 
\centering
\label{scen1}
\includegraphics[scale=0.3]{figures/conv_nonuc}
\caption{Conventional technologies, no new nuclear.}
\end{figure}

\begin{figure}[h] 
\centering
\label{scen2}
\includegraphics[scale=0.3]{figures/conv_nuc}
\caption{Conventional technologies, with new nuclear.}
\end{figure}

\begin{figure}[h] 
\centering
\label{scen3}
\includegraphics[scale=0.3]{figures/newtechs_nonuc}
\caption{Emerging technologies, no new nuclear.}
\end{figure}

\begin{figure}[h] 
\centering
\label{scen4}
\includegraphics[scale=0.3]{figures/newtechs_nuc}
\caption{Emerging technologies, no new nuclear.}
\end{figure}

\subsection{Sensitivity analysis}