\section{Background and literature review} \label{litreview}
The Paris Agreement commits individual nations to significant carbon reduction over time through the Intended Nationally Determined Contribution (INDC) mechanism (UNFCCC, 2019). Japan, as a signatory to the Paris Agreements has submitted an INDC with the following goals and timelines: by 2030, to reduce GHG emissions by 26\% compared to 2013 levels, and by 2050 to reduce overall GHG emissions by 80\% or more, through the “development and diffusion of low-carbon technologies and transition to a low-carbon socio-economic structure” (UNFCCC, 2015). Cognizant of these targets, a numfgber of authors have evaluated Japan,s future energy system using a variety of modeling approaches. Using the MARKAL model, considering the uncertainties of technology development, Ozawa et al., found that hydrogen will play a major role in the future energy system, reliant on both nuclear power and carbon capture and storage (CCS) to reduce electricity sector emissions to nearly zero by 2050 (2018). Recognizing the benefits that renewable energy will play in reducing carbon emissions, and the issues of intermittency, Li et al. explored the role of hydrogen as a storage medium through power to gas (P2G) approaches in Kyushu, Japan. Their study identified that P2G can increase the effective utilization rate of renewable energy, and the use of hydrogen in the gas network, effectively pairing the electricity and gas networks, overcomes current renewable electricity curtailment issues (2019). Cognizant of the Japanese government,s strategic approach to carbon reduction out to 2050 via energy system reform, Chapman & Pambudi also identify a strong role for nuclear, renewables and hydrogen under a carbon constrained, optimal cost MARKAL/TIMES simulation approach (2018). Considering economic conditions and demographic trends such as moderate GDP growth and rapid aging which are occurring in Japan, Kuriyama et al., suggest that 2030 targets can be met or exceeded (i.e. up to 42\% GHG reduction) with limited renewable energy growth at a 15\% contribution from nuclear, or even without nuclear, under a renewable growth scenario (2019). It appears that these trends and energy system changes will be insufficient to meet the more ambitious 2050 targets. Taking a more holistic view, in line with the Japanese government,s 3E+S targets, ambitious research and development to enable high levels of renewable deployment appears to be necessary to not only meet deep emission reduction goals, but to also reduce Japanese dependence on imported fuels, cognizant of both CCS and nuclear deployment rates into the future. Consensus on policy options and priorities also have a large influence on modeling outcomes for the  Low Carbon Navigator assessing Japanese energy and emission options out to 2050 (Moinuddin & Kuriyama, 2019). A seminal work by Sugiyama et al., brings together a number of modeling approaches for Japan,s long term (i.e. to the year 2050) climate change mitigation options, utilizing national and global general and partial equilibrium models (2019). Model results are contrasted under six scenarios which incorporate a baseline and a range of emissions reductions (50-80\%) and regional obligations for global models (ibid.). under the Paris Agreement target of an 80\% reduction, each of the models assessed recognize the importance of renewable energy deployment by 2050, notably hydro, solar and wind, with varying contributions from nuclear energy and fossil fuels, predominantly natural gas. Additionally, for Japan, the option for import of carbon-free hydrogen was identified as potentially playing a critical role (Sato, 2005; Akimoto et al., 2010; Matsuo et al., 2013; Fujii and Komiyama, 2015; Oshiro & Masui, 2015; Sugiyama et al., 2019). Many studies consider hydrogen a critical part of Japan,s low-carbon energy transition, specifically toward improving energy security, its ability to be produced from multiple sources and lack of emissions when combusted (Iida & Sakata, 2019). Global modelling efforts consider the incorporation of long-distance international transport of hydrogen with end uses dominated by fuel cell passenger and freight vehicles and power generation, via both mixed and direct combustion. Electricity from hydrogen is estimated to emerge in Japan from 2030 onwards, as nuclear and coal fired power generation reduce toward 2050 (Ishimoto et al., 2017). From a policy standpoint, Japan has committed to achieving a hydrogen society with the primary goal of cost parity of hydrogen with competing fuels, requiring a three-fold reduction in cost by 2030, and further reductions into the future (Nagashima, 2018). Under the Basic Hydrogen Strategy, the Japanese government aims to realize low cost hydrogen use in power generation, mobility and industry, develop international supply chains to ensure stable supply, expand renewable deployment and revitalize regional areas and develop hydrogen related technologies (METI, 2017). The strategy aims to account for both economic and geopolitical impacts and the need to prioritize research and development to overcome the economic and technical challenges (Nagashima, 2018). A common thread across previous research is that the uncertainty surrounding carbon capture and storage (particularly with regard to scaling up and public acceptance issues) and the role that nuclear energy will play, largely due to policy reform which occurred in response to the nuclear accident in 2011 (Oshiro et al., 2019). 
The model and approach proposed in this research is unique, as it builds on the works and modeling consensus outlined in the literature review and expands the consideration of technologies beyond conventional and recent technologies to also include emerging and disruptive technologies post-2050. By modeling the Japanese energy system out to the year 2100, our aim is to detail the mid and long-term impacts of technological development and market penetration, and to identify the suite of technologies which could underpin the successful achievement of carbon reduction goals against the backdrop of a long time-scale that incorporates the effects of the retirement of existing fossil fuel, nuclear and renewable generation capacity. This work also leverages the dynamic simulation capabilities of TIMES <CITE LOULOU> by incorporating time varying trends for economic parameters such as investment and o&m costs, efficiency, deployment dates, scalability, growth rates, and emission coefficients. Our model also does not neglect life-cycle emissions from nuclear, renewables or any of the emerging CCS or hydrogen technologies, as the emissions associated with the construction of these technologies, while insignificant compared to fossil fuels, are significant when considering emission levels associated with an 80\% emission reduction. This also allows for a more meaningful analysis when assessing the carbon neutrality of the energy mix, and the effect of a carbon tax.
