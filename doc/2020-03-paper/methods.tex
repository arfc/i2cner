\section{Methodology} \label{method}
% what is TIMES
\subsection{TIMES Model Description}
\gls{TIMES} models dynamic energy systems and simulates transition scenarios as a mixed-integer linear optimization problem that is subject to a primary objective function and additional constraints. The generation, refinement, supply, storage, and trade of energy commodities across multiple sectors and multiple regions are modeled using a wide variety of in-built commodity and process types. Emissions can be associated with energy commodities or processes as an emission coefficient per unit commodity produced or consumed. 

%basic features of model
The objective function in our single-region model is the overall cost of the transition. The major constrains in our simulations are the demand for electricity (see Table \ref{demand}), emission constraints on the electricity-generation sector based on Japan's \gls{INDC} (see Table \ref{co2-limits}), and feasible nameplate capacity deployment limits (see Table \ref{caplim}). Miscellaneous assumptions are summarized in table \ref{misc-assump}. Therefore, our model minimizes the transition cost, while meeting the increasing electricity demand and achieving the required emission cuts using a combination of generation and storage technologies. 

While electricity demand in the near future is expected to grow \cite{noauthor_electricity_2019}, long term electricity demand in Japan is expected to plateau, or even decrease, due to Japan's aging population. However, precisely quantifying this rate of decrease is challenging as there is potential for increased electrification of  transportation and industrial sectors. Hence, post-2030, we have assumed a demand curve based on the likelihood of increased electrification driving increasing demand, but the demand eventually plateaus due to the aforementioned expected demographic changes. The unique initial condition of the post-Fukushima Japanese electricity supply system is captured using \gls{EDMC} data from 2013-2016 \cite{the_institute_of_energy_economics_japan_energy_2018}. Long term impacts of factors such as the retirement of the existing nuclear reactor fleet, and the deployment of emerging technology is assessed by simulating the system until 2100. The carbon cost of each technology is accounted for using an emission coefficient that incorporates both direct emissions and life cycle emissions averaged over the entire operating lifetime for every technology, as applicable. The daily and seasonal variability of renewables is incorporated using \gls{TIMES} day-night and seasonal time periods \cite{loulou_etsap-tiam_2008}. The availability of renewables varies during these time periods based on the annually averaged capacity factors of renewables in Japan \cite{the_institute_of_energy_economics_japan_energy_2018} cite IRENA.

\begin{table}[!ht]
	\caption{Demand increase over time.}
	\vspace{0.1in}
	\begin{tabularx}{\textwidth}{p{0.5\textwidth} p{0.5\textwidth}}
		\hline
\textbf{Year} & \textbf{Annual demand increase} \\
\hline
2017-2030 & 1.7 \% \cite{noauthor_electricity_2019} \\
2031-2050 & 1.0 \% \\
2051-2070 & 0.5 \% \\
2070-2100 & 0.0 \% \\
\hline 
	\end{tabularx}
\label{demand}
\end{table}

\begin{table}[!ht]
	\caption{CO$_2$ constraints.}
	\vspace{0.1in}
	\begin{tabularx}{\textwidth}{p{0.1\textwidth} p{0.22\textwidth}p{0.16\textwidth} p{0.4\textwidth}}
		\hline
\textbf{Year} & \textbf{Emission limit} & \textbf{Base Year} & \textbf{Reduction from base year} \\
\hline
2030 & 438 Mt CO$_2$-eq. & 2013 & 26 \% \\
2050 & 75 Mt CO$_2$-eq. & 1990 & 80 \% \\
2100 & 75 Mt CO$_2$-eq. & 2050 & 0 \% \\
\hline 
	\end{tabularx}
\label{co2-limits}
\end{table}

%scenario description
To explore possible pathways to curbing \gls{GHG} emissions, we simulated a total of five transition scenarios of varying likelihood, with different sets of technologies enabled for deployment, as described in Table \ref{scen-table}. The first set includes conventional technologies such as  \gls{USC},\gls{lng}, solar photovoltaic, wind energy (with onshore, offshore-fixed and offshore-floating considered separately) and utility-scale lithium-ion battery storage. New deployments of oil-fuelled power plants are neglected due to the declining use of oil for electricity generation in accordance with Japan's goal of energy security and independence, as per the Basic Energy Plan. The second set of technologies considered includes emerging carbon-neutral technologies that are already commercialized or close to commercialization, namely emerging solar photovoltaic (representative of technologies such as perovskites, CdTe),\gls{CCS}, and utility-scale hydrogen power. For hydrogen power, steam reforming, steam reforming with \gls{CCS}, \gls{AEC}s,\gls{PEMEC}s,\gls{PEMFC}s, and \gls{SOFC}s were incoporated based on their technological potential. Along with these two technology groups, we also explore the potential impact of nuclear energy. Nuclear power has significant advantages over renewables due to its long operational lifetimes, and consequently, extremely low life-cycle emissions, and a high capacity factor. However, nuclear power faces extremely low public acceptance in Japan after the Fukushima Daiichi accident, and its future in Japan is highly uncertain. Hence, transition scenarios with and without new nuclear reactor deployment must be juxtaposed to assess the importance of the role of nuclear in emission reduction. Finally, the long-term impact of nascent hydrogen technologies on the hydrogen economy is assessed in an additional scenario. In this scenario, the potential commercialization of \gls{SOEC} and \gls{PWS} post-2050 is explored in the absence of new nuclear power.

%misc assumptions
Exogenous variables such as economic data, emission coefficients, nameplate capacity limits, and growth rates are detailed in tables \ref{eco}, \ref{caplim}, and \ref{growrate} respectively. Prices and projections for fossil fuels and nuclear fuel were incorporated \cite{wittenstein_projected_2015, world_bank_commodity_2016, international_energy_agency_world_2019}. Learning curves for costs and life-cycle emissions are compiled from existing data based on expected scaling of manufacturing, availability of manufacturing materials, and the use of clean energy for manufacturing energy system components. These learning curves are modelled as linear functions interpolated between the data values used, with the curve plateauing at the latest value for a given parameter, as detailed in Table \ref{eco}. Capacity limits of renewables and \gls{PWS} are based on their land-use requirements. The maximum annual capacity growth rates for existing technologies are held constant. The growth rate of nuclear power is based on historic trends and current pressure vessel manufacturing limitations \cite{iaea_pris_nodate}. The reactor size assumed in this study is 1165 MWe. Due to a projected increase in the share of renewables, nuclear power plants must be able to load follow to a certain extent, which is simulated in a limited way in our model based on French reactors' range of capacity factors. The growth rates of all emerging technologies are modeled on the rates observed for solar photovoltaic technology, with rapid initial growth followed by gradual reduction, eventually reaching a moderate maximum attainable growth rate. One notable exception is the maximum growth rate of emerging solar technologies, which we have assumed to be the same as that of existing solar photovoltaic technologies. We believe that these technologies, some of which are already commercialized or close to commercialization, will benefit immensely from the already streamlined solar photovoltaic manufacturing and supply chain. Therefore, they could be deployed as rapidly as conventional solar photovoltaic. 

All hydrogen storage devices are operated with a maximum availability factor of 90\%, making them extremely flexible for load following. Long-term storage of hydrogen is also available using hydrogen tanks with appropriate loss factors \cite{iea_technology_2015}. For hydrogen electrolyzers and fuel cells, life-cycle emissions from just the stack are considered, as \gls{BOP} emissions from utility scale hydrogen depend strongly on the type of plant and the source of energy used for electrolysis. Our assumptions about the reduction in the investment costs and life-cycle emissions of batteries are conservative, due to the rising cost of cobalt and nickel, and lithium-ion manufacturing being concentrated in high \gls{GHG}-emitting nations, respectively \cite{oliveira_environmental_2015,emilsson_lithium-ion_2019,turcheniuk_ten_2018,simon_potential_2015}. 

\begin{table}[!ht]
	\caption{Scenario definition.}
	\vspace{0.1in}
	\begin{tabularx}{\textwidth}{p{0.15\textwidth} p{0.25\textwidth} p{0.25\textwidth} p{0.35\textwidth}}
\hline 
\textbf{Scenario}& \textbf{Emerging tech.} & \textbf{New nuclear} & \textbf{Nascent tech.}\\
                 & \textbf{enabled} & \textbf{enabled} & \textbf{enabled}\\
                  \hline
%1               &   \xmark       &      \greencheck     \\ 
%2               & \xmark       &         \xmark       \\ 
%3               &   \greencheck     &      \greencheck     \\ 
%4               &   \greencheck     &         \xmark       \\
1               &  No       &         No     &     No  \\ 
2               &   No       &      Yes     &     No  \\ 
3               &   Yes     &         No      &     No   \\
4               &   Yes     &      Yes     &     No  \\ 
5               &   Yes     &      Yes     &     Yes  \\ 
\hline
	\end{tabularx}
\label{scen-table}
\end{table}



\subsection{Sensitivity analysis}
%approach, goal, what are we hoping to learn

