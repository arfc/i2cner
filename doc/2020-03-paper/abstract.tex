\begin{abstract}

We simulated possible pathways to meeting 2030 and 2050 emission targets within the Japanese electricity supply sector using a single-region \gls{TIMES} model. Key features of our simulations include the incorporation of novel technologies, like hydrogen electrolysers, carbon capture, photochemical water splitting, and emerging photovoltaic cells, long-term impact assessment up to the year 2100, the inclusion of life-cycle emission and learning curves for technology costs and emission coefficients. Results indicate that a hybrid approach, using nuclear power and hydrogen from renewable energy-based electrolysis, is cost-effective and provides long-term emission reduction along with energy security. Nuclear, wind, solar, and hydrogen from renewables emerge as key emission reduction technologies, while natural gas with carbon capture plays a minor role.

\end{abstract}
