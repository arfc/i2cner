\section{Discussion}
% This should explore the significance of the results of the work, not repeat
% them. A combined Results and Discussion section is often appropriate. Avoid
% extensive citations and discussion of published literature.

%Notes for Ansh:
%Importance of nuclear : cost of various transitions, flexibility by itself and when coupled with storage (Li-ion, H2). How it precludes the need for CCS.
%Urgency of transition : scenarios + sensitivity analysis.
%Li-ion storage: land use requirements for each case, qualitatively discuss Co, Ni requirements. Say Li is okay.
%AECs,PEMECs,SOECs need to be highly flexible with high availability factors.
%SOFCs: need for extremely responsive SOFCs with high availability factor, flexible operation (cite challenges from NREL).
%pemecs still part of cost optimal solutions
%identify key techs, same fcs needed always
%nuclear is key
%pws will have to be cheaper/cost competitive with electrolyzers to have a role
%main H2, role of CCS LNG - secondary alternative to H2 - as shown by correlation with SOFC inv cost. Not good on its own. Supplement to H2.
%H2 non trivial despite large range of perturbation
% FC flexibility, more storage/H2 less flex for CCS, nuclear

%urgency
As the results of all base case scenarios indicate, rapid retirement of fossil fuels and deployment of carbon neutral technologies is urgently required to achieve emission reduction targets in Japan. All coal and oil plants need to be shut down between 2025-2030, and natural gas use must be reduced dramatically. It is also imperative that the existing nuclear reactors be operated at full capacity by 2022. Since hydrogen plays a key role in zero to moderate nuclear deployment scenarios (Figs. \ref{scen3} and \ref{scen4}), it appears prudent to invest in hydrogen power. As the model uses learning curves for technology prices and our objective function is transition cost, hydrogen technologies are deployed as late as possible in order to minimize costs by utilizing technologies at their cheapest, while there is still time to deploy them and achieve emission goals. Therefore,  results indicate that it is imperative that Japan deploy an increasing amount of renewables including wind power between 2020-2030, and be prepared to deploy and rapidly scale up hydrogen power by 2030 at the latest in the absence of new nuclear, or by 2035 with new nuclear.

%nuclear
The scenario with the largest nuclear deployment, Scenario 2, also has the lowest transition cost. When used with hydrogen, 50 MW of new nuclear results in the greatest reduction in emissions after 2050 out of all 5 scenarios. It also reduces the cost of the transition from 3.18 Trillion USD in Scenario 3 to 2.80 Trillion USD in Scenario 4, a difference of 12\% of the system cost. The estimated savings in cost and emissions due to nuclear are conservative, as we have not incorporated the cost of transporting hydrogen utilizing tankers or pipelines. The large emission cuts achieved with the help of nuclear also allow natural gas to continue operating until 2100, while meeting emission goals. As natural gas is well suited to peaking and load-following, when coupled with flexible fuel cell technologies, it could engender in an extremely stable energy system. 

For the transitioning energy system in Japan, under all scenarios investigated, nuclear plays a key role in reducing emissions. A potential strategy for Japan could include the reinvigoration of its nuclear energy sector by restarting existing plants, and if politically feasible, constructing new plants, and investing in advanced reactor research. To minimize the environmental impact from life cycle emissions, it is necessary that all existing and any new reactors be operated for a lifetime of 60 years or more at a high capacity factor. Premature decommissioning due to operational problems or lack of public acceptance need to be avoided. Therefore, prioritizing reactor safety for resilience to disasters and in order to regain the public's trust in nuclear power, and increasing public awareness of nuclear's vital role in mitigating carbon emission is key to achieving 2030 and 2050 emission targets.

%renewables and flexibility
In all scenarios, solar and wind supply a large portion of the electricity demand. Unless large numbers of nuclear power plants are constructed, Japan will need to invest heavily in offshore wind farms by 2030 (Figs. \ref{scen1},\ref{scen3}-\ref{scen5}). If hydrogen and CCS are not deployed by 2035, investment in offshore floating turbines may also become necessary (Fig. \ref{scen1}). Under a scenario with a relatively large share of renewables, grid flexibility becomes extremely important. Such a scenario requires significant investment in storage technologies, and ensuring that all emerging technologies are flexible and responsive. As natural gas is the only extant option for load-following, it is necessary to invest in nuclear or hydrogen to achieve emission cuts while simultaneously keeping natural gas operational for load-following. Additionally, any nuclear power plants and fossil fuel power plants which use CCS must be able to load-follow, or be coupled to storage, ideally electrolysers, in order to store excess electricity for peak demand. Any fuel cell technology utilized must also be extremely flexible and responsive. Since the model prioritizes \gls{SOFC}s over \gls{PEMFC}s due to their higher efficiency, it is important that \gls{SOFC}s reduce their startup times to have an edge over \gls{PEMFC}s in utility-scale applications.

%batteries
Lithium-ion is the preferred storage medium in the absence of hydrogen. However, as the base case scenarios indicate, extremely large capacities of storage must be deployed to achieve a stable grid that can sustain the large capacities of renewables required for deep emission cuts. While lithium may be available, cobalt and manganese reserves are limited \cite{scrosati_lithium-ion_2011,simon_potential_2015,turcheniuk_ten_2018} , which may inflate the prices of lithium-ion storage if it is relied upon as the primary storage medium. It may be preferable to redesign batteries to reduce the amount of rare minerals used in their manufacturing, and improve recycling to increase the recovery rate of rare metals. The life cycle emissions from batteries must also be reduced by using cleaner materials and electricity for manufacturing. If hydrogen storage is available, battery storage storage serves as a near-term transition technology after which hydrogen storage dominates (Figs. \ref{scen3}-\ref{scen5}).

%hydrogen - key players, key parameters, path forward?
After nuclear, hydrogen power emerges as the second most efficacious technology for achieving 2050 emission goals. As seen in the sensitivity analysis results, hydrogen maintains a significant share of the electricity mix despite a wide range of perturbations to key parameters of the hydrogen sector. While \gls{SOFC}s and \gls{SOEC}s are preferred over \gls{PEMFC}s and \gls{PEMEC}s in all of our simulations, the role of hydrogen is so important that if \gls{SOFC} or \gls{SOEC} deployment is disabled, the model replaces them with inferior \gls{PEMFC}s or \gls{PEMEC}s respectively to recreate similar energy mixes. Key technologies that aid in decarbonization in our simulations are \gls{SOFC}s and \gls{PEMEC}s, which are close to maturation. Steam reforming, with or without CCS, does not get utilized in our simulations due to its high emission coefficient. At the lower end of the technology readiness scale, \gls{SOEC}s emerge as tremendously disruptive due to their high efficiency. However, their operational lifetimes need to be increased significantly, life cycle emissions must be kept low, and cost-competitiveness with \gls{PEMEC}s must be achieved in order to realize their potential. \gls{PWS} plays a marginal role as it is not cost-competitive with electrolysis. From the simulation output data, the investment cost of \gls{SOFC}s emerges as a critical parameter. Therefore it is vital to reduce fuel cell investment costs to make deep emission reduction economically feasible. However, low response times and high availability are two other key factors that are implicit in our assumptions. If \gls{SOFC}s are not as flexible as assumed, they are replaced in our simulations by \gls{PEMFC}s, which are known to be more flexible and closer to commercialization. In Scenarios 3 and 4, solar, wind, and nuclear are used to generate hydrogen. Therefore it would be economically favourable to couple utility-scale renewables and nuclear power plants with electrolysis plants. Waste heat from nuclear could also be used to produce hydrogen, and would reduce reliance on renewables for electrolysis. As mentioned previously, more flexible and responsive fuel cells and electrolyzers are likely to be the dominant technology in such a system.

%ccs
CCS plays a small role in our base case scenarios as a transitional technology, despite being cheaper than hydrogen. Its share is found to be largely insensitive to its investment costs as evidenced by the sensitivity analysis. This is mainly due to its large emission coefficient. Marginal gains are expected in its penetration if the efficiency of CCS is increased to reduce its direct emissions. Reducing indirect emissions is likely to have a greater impact. This could be achieved in the future by increased electrification of the industrial sector, and the use of hydrogen to produce steel.

Although our analysis relies on optimal solutions and scenarios and is technologically agnostic, policy makers in Japan will face difficult future decisions as to the long-term nature of the low-carbon energy system. One approach could be to prioritize nuclear energy and to reap the benefits of inexpensive, low-carbon energy, at the cost of achieving consensus from the public for its long-term deployment and use. Another low carbon energy pathway proposed by our results is a transition toward a hydrogen economy, utilizing hydrogen as a storage medium to engender significant deployment of renewable energy. The hydrogen pathway also provides more energy security, obviating the use of coal and oil and mitigating the use of natural gas. As social opposition to nuclear is a prominent issue in Japan, the deployment of a parallel nuclear and hydrogen-based energy system is unlikely. However, our results indicate that these two approaches are not mutually exclusive. The cost of transitioning to a hydrogen energy system without deploying any new nuclear is much higher than that of any of the nuclear-inclusive options. Deploying nuclear in tandem with hydrogen provides a cost-effective compromise. An approach reliant on a mature technology like nuclear also improves the likelihood of Japan meeting its 2050 emission goals, as at the time of writing, the timely success of the Japanese hydrogen plan is far from certain. At the very least, as is the case for CCS and fossil fuels, nuclear power may offer a "bridging" option, providing a low-carbon pathway in the short term through extended nuclear lifetimes and limited new builds, allowing sufficient time for the maturation of hydrogen and renewable-based energy options for longer-term deployment timelines. If supported politically, long-term use of nuclear power could provide emission cuts well beyond the 2050 targets. Policy which is cognizant of these economic, environmental, and social aspects of energy systems is required to deliver a low-cost, low-carbon energy future for Japan which is socially acceptable.