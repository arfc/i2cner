\documentclass[14pt,a4paper]{article} %openany
\usepackage[affil-it]{authblk}
\usepackage[english]{babel}
\usepackage{graphicx}
\usepackage{rotating}
%\usepackage{bibtex}
%\usepackage[utf8]{inputenc}
%\usepackage[english]{babel}
\usepackage{adjustbox}
\usepackage{courier}
\usepackage{verbatim}
\usepackage{url}
\usepackage{hyperref}
\usepackage{float}
\usepackage{array}
\usepackage{breakcites}
\usepackage{gensymb}
%\usepackage[backend=biber]{biblatex}
\usepackage{booktabs,tabularx}
\usepackage{listings}
\usepackage{appendix}
\usepackage{cite}
\usepackage{blindtext}
\usepackage[utf8]{inputenc} % Required for inputting international characters
\usepackage[T1]{fontenc} % Output font encoding for international characters
\usepackage{mathpazo} % Palatino font
\usepackage{graphicx} % For the logo
\usepackage[acronym,toc]{glossaries}
%\newacronym{<++>}{<++>}{<++>}
%\newacronym{<++>}{<++>}{<++>}
\newacronym[longplural={metric tons of heavy metal}]{MTHM}{MTHM}{metric ton of heavy metal}
\newacronym{EDMC}{EDMC}{Energy and Data Modelling Center}
\newacronym{FEPC}{FEPC}{Federation of Electric Power Companies of Japan}
\newacronym{EIA}{EIA}{Energy Information Administration}
\newacronym{TIMES}{TIMES}{The Integrated Markal-EFOM System}
\newacronym{I$^2$CNER}{I$^2$CNER}{International Institute for Carbon-Neutral Energy Research}
\newacronym{CCS}{CCS}{Carbon Capture and Sequestration}
\newacronym{JWPA}{JWPA}{Japan Wind Power Association}
\newacronym{LCoE}{LCoE}{Levelized Cost of Electricity}
\newacronym{ABWR}{ABWR}{Advanced Boiling Water Reactors}
\newacronym{MoE}{MoE}{Ministry of Environment}
\newacronym{METI}{METI}{Ministry of Environment}{Ministry of Economy, Trade and Industry}
\newacronym{DSCINV}{DSCINV}{Discrete Investment}
\newacronym{C2Peak}{C2Peak}{Contribution to Peak}
\newacronym{MaxCAP}{MaxCAP}{Maximum Capacity}
%\acro{H2O}[$\mathrm{H_2O}$]{water}



%\bibliographystyle{numeric}

\begin{document}

%----------------------------------------------------------------------------------------
%    TITLE PAGE
%----------------------------------------------------------------------------------------

\begin{titlepage} % Suppresses displaying the page number on the title page and the subsequent page counts as page 1
    \newcommand{\HRule}{\rule{\linewidth}{0.5mm}} % Defines a new command for horizontal lines, change thickness here
    
    \center % Centre everything on the page

    %------------------------------------------------
    %    Title
    %------------------------------------------------
    
    \HRule\\[0.2cm]
    
     \begin{minipage}{0.4\textwidth}
        \includegraphics[width=\textwidth]{arfc-logo}
        \end{minipage}%
        \begin{minipage}{0.6\textwidth}
        {\begin{flushright}\huge\bfseries Dynamic Transition Analysis With \gls{TIMES} \end{flushright}}
        {\begin{flushright}\large\textit{\gls{I2CNER}  Project Report}\end{flushright}}

        \end{minipage}

    \vspace{0.2cm}
    \HRule
    \vspace{0.5cm}
    
    %------------------------------------------------
    %    Author(s)
    %------------------------------------------------
    
    \begin{minipage}{0.4\textwidth}
        \begin{flushleft}
            \large
            \textit{Author}\\
            Anshuman \textsc{Chaube}\\
        \end{flushleft}
    \end{minipage}
    ~
    \begin{minipage}{0.4\textwidth}
        \begin{flushright}
            \large
            \textit{Principal Investigator}\\
            Kathryn D. \textsc{Huff} % Supervisor's name
        \end{flushright}
    \end{minipage}
    
    % If you don't want a supervisor, uncomment the two lines below and comment the code above
    %{\large\textit{Author}}\\
    %John \textsc{Smith} % Your name

    %------------------------------------------------
    %    Report Number
    %------------------------------------------------
    \vspace{1cm}
    \textsc{\LARGE\bfseries UIUC-ARFC-2018-00} \textit{work in progress} % Replace YYYY with the year, NN with report index
    \vspace{0.5cm}
    
    %------------------------------------------------
    %    Date
    %------------------------------------------------
    
    \vspace{0.5cm} % Position the date further down the remaining page
    {\large\today} % Date, change the \today to a set date if you want to be precise
    \vspace{0.5cm}

%------------------------------------------------
    %    Headings
    %------------------------------------------------
    
    \textsc{\LARGE Advanced Reactors and Fuel Cycles}\\[0.25cm] % Research Group
    
    \textsc{\large Dept. of Nuclear, Plasma, \& Radiological Engineering}\\% Department
    
    \textsc{\large University of Illiois at Urbana-Champaign}\\ % University


    
    %------------------------------------------------
    %    Logo
    %------------------------------------------------
    
    
    \vspace{0.5cm}
    \includegraphics[scale=0.2]{arfc-smol}
    \includegraphics[scale=0.3]{i2cner-logo}
    \includegraphics[scale=0.2]{wpi-logo}
    \includegraphics[scale=0.04]{ku-logo}\\[1cm] % Include a department/university logo - this will require the graphicx package
     
    %----------------------------------------------------------------------------------------

    %------------------------------------------------
    %   Funding 
    %------------------------------------------------
    % For this section, either use \vfill to fill the space 
    % or insert funding acknowledgement
    \textit{The authors gratefully acknowledge the support of the International Institute for Carbon
Neutral Energy Research (WPI-I2CNER), sponsored by the Japanese Ministry of Education, Culture, Sports, Science and Technology.}  

\end{titlepage}

\section{Introduction}
We initiated a project in January 2018 to simulate dynamic transition scenarios for the energy industry in Japan to suggest pathways for minimizing carbon emissions. This report is a summary of the progress we have made so far, the challenges we currently face, and the future direction of this research. 

\section{Progress Summary}

The tasks that we performed can be divided into two categories: technical tasks associated with implementation of details and features in our model, and data collection and organization. Our accomplishments have been:

\begin{itemize}

\item \textbf{Installation of and familiarization with VEDA} (January – March 2018) : To model Japan's energy industry, we chose VEDA, a \gls{TIMES} \cite{loulou_documentation_2005} \cite{gargiulo_documentation_2005} generator. We found the format of the developer-prescribed model files restrictive and unsuitable for our purposes. Therefore, we took the time to develop our own model files, which we have progressively refined since then.\\

At the same time, we collected data pertaining to electricity generation and carbon emissions.

\item \textbf{Incorporation of fossil fuel-related data} (April – May 2018): We incorporated data for electricity generation from fossil fuels from the \gls{EDMC} databank\cite{noauthor_energy_2018}, along with creating a simplified demand process reflecting the recent trends in electricity demand in Japan. \\

While collecting this data, we noticed that the \gls{EDMC} databank that we have been relying on has no data for the amount of electricity generated from individual fossil fuels for the years 2011-12. Instead, the amount of electricity generated from coal, oil, and natural gas is lumped together in one category titled "thermal". Further, the 2016 data seems slightly inconsistent across different data tables in the \gls{EDMC} databank. The source of variation in these numbers is likely to be the changes in the electricity distribution system of Japan since 2016.

\item \textbf{Incorporation of nuclear, hydropower and renewables into the model} (June – August 2018): The process of incorporating these into the model was similar to the previous energy sources, but simpler, since the data obtained for these energy sources from \gls{EDMC} was consistent across \gls{EDMC} data tables and secondary sources \cite{noauthor_energy_2018} \cite{noauthor_iea_2017} \cite{noauthor_japan_2017}. We have also included processes for the projected growth of nuclear, solar and wind based on data from various studies, reports and articles. \cite{publicover_japan_2017} \cite{dincer_analysis_2011} \cite{noauthor_geothermal_2018} \cite{heger_wind_2016} \cite{noauthor_operational_2013} \cite{noauthor_electricity_2017}

\item \textbf{Refining CO\textsubscript{2} emission processes} (August – September 2018) : While we had been modelling CO\textsubscript{2} emission processes in parallel with the electricity processes, it was only after incorporating all conventional energy sources that we could move on to fine-tuning CO\textsubscript{2} emission values to ensure they match the actual emissions from Japan. The major obstacle we faced was the absence of data pertaining to electricity generation from individual fossil fuels, with each fossil fuel's energy cycle having different emission coefficients. We estimated the missing figures based on previous years' trends \cite{noauthor_energy_2018} \cite{noauthor_national_2018} and arrived at reasonable estimates of electricity generation, which result in CO\textsubscript{2} emission values that differ from actual values by about 5\% at most.

\item \textbf{October 2018 – January 2019:}

\begin{enumerate}

\item \textbf{Changing simulation timeframe to 2013-2100:} As discussed previously, it became impossible to find exact data for fossil fuels for the years 2010,2011 and 2012. Hence the total CO$_2$ emissions for those years were very slightly off the mark. We sidestepped this problem by changing the initial year to 2013, for which we have exact data from \gls{EDMC} \cite{noauthor_energy_2018}.\\

\item \textbf{Incorporation of the \gls{PEAK} factor \cite{gargiulo_documentation_2005} factor:} This parameter is defined as the fraction of a resource's capacity that is guaranteed to be available during peak demand. This introduces a notion of an energy resource's reliability. Its incorporation reduced excessive deployment of wind and solar. The \gls{PEAK} factor values in the model \cite{kato_energy_2016} do not take into account the true daily or seasonal variation of wind and solar, as the factor is annually averaged.\\

\item \textbf{Basic \gls{CCS} Implementation} : Some \gls{CCS} data \cite{kato_energy_2016}  was incorporated into \href{https://github.com/arfc/i2cner/tree/master/JPN-Main-Model/active/i2cner-nonuc}{one of the models}. However, no \gls{CCS} gets deployed in our models. We believe  it should be deployed for an intermediate time-period, since in the absence of nuclear, only \gls{CCS} can provide reliable, clean energy in conjunction with renewables. We have identified a few shortcomings in our model, some of which contribute to this problem:

\begin{enumerate}

\item Large amounts of offshore wind can be deployed. While we were initially reluctant to hard-code strict capacity limits into our model, we have since realized that Japan's underdeveloped offshore wind industry will not reach its full potential for a very long time, as offshore wind is extremely expensive to deploy due to the unusually deep seabed that is very close to the Japanese coast. \gls{JWPA} projections \cite{heger_wind_2016} are already rather ambitious, and our models should be more closely aligned with them, allowing for at-most a 10-20 \% increase in deployment capacities.

\item Wind and solar are treated as any other energy source, with their daily or seasonal variance not truly taken into account (discussed below). Their installed capacity should be matched by storage or natural gas. Possible ways to implement this are discussed later.

\item We may be overestimating \gls{CCS} costs. The costs associated with \gls{CCS} for Japan have been hard to find as Japan, instead of building \gls{CCS} pipelines like the US or China, intends to build a shipping network for offshore storage of captured and compressed CO$_2$. While this would make \gls{CCS} plants more expensive in Japan, we cannot arrive at an exact figure. Based on our interaction with our Energy Analysis Division colleagues at Kyushu university, the costs of this are still being explored by the Japanese government.

\end{enumerate}

\item \textbf{February 2019 – April 2019:}

\begin{enumerate}

\item \textbf{Gradual collection and replacement of \gls{LCoE} data with accurate cost structure:} The results presented at the \gls{I2CNER} Annual \gls{EAD} workshop \cite{chaube_dynamic_2019} were based entirely on \gls{LCoE} analysis, as \gls{LCoE} data and projections were readily available. However, it is more suitable to incorporate cost data in the recommended \gls{TIMES} format, that is with the investment/capital cost, and fixed and variable \gls{OM} costs. This is due to the fact that when there is no investment cost associated with an energy source, the deployment or premature retirement has no cost-penalty. This causes resource deployments for unreasonably brief periods (see fig. \ref{flatgood} ).

\item \textbf{Incorporation of semi-discrete investment sizes:} Discrete investment sizes were incorporated in most scenarios' \gls{DSCINV} files\cite{gargiulo_documentation_2005}, whereas the slightly improved semi-discrete capacity sizes are incorporated in the \href{https://github.com/arfc/i2cner/tree/master/JPN-Main-Model/active/co2-constrnt-conv-nonuc}{conventional-no-nuclear model}. It is desired that all \gls{DSCINV} files in the remaining three models include a similar semi-discrete capacity installation structure, as this helps eliminate the production-exceeding-demand bug (see fig. \ref{flatbug} and fig. \ref{flatgood} ).\\

\end{enumerate}


\end{enumerate}


\end{itemize}

\section{Model description}

\subsection{Assumptions}

	Our model focuses on the electricity generation sector. The following assumptions and limitations are present in our model:
	
\begin{enumerate}

\item All the energy generated by a given process is transferred to the grid without losses. Since the \gls{EDMC} data has values in terms of units of electrical energy produced (GWh), we have had no need for incorporating data about raw fossil fuel consumption, plant efficiency, and utilization factors for the initially deployed electricity generation sources.

\item \gls{LCoE} for fossil fuels has been held constant throughout the simulation \cite{chapman_energy_2018} \cite{noauthor_lazards_2017} \cite{noauthor_iea_2017}. \gls{LCoE} projections for wind and solar have been incorporated \cite{noauthor_lazards_2017}.

\item Oil-based electricity is retired relatively quickly, and new oil-based electricity deployment is disabled, due to the emphasis of the Japanese government on energy self-sufficiency and minimizing costs, and due to a general trend in the \gls{EDMC} data \cite{noauthor_energy_2018} indicating declining use of oil.

\item Nuclear capacity is increased in chunks equivalent to the capacity of GE-Hitachi's \gls{ABWR} \cite{ge_advanced_2007}, which are under consideration for construction \cite{noauthor_electricity_2017}.

\item Solar – Any new solar capacity created by the model has been assumed to be non-tracking type.

\item Hydropower – held constant at current levels.

\item Geothermal is expanded to its maximum potential \cite{noauthor_geothermal_2018}.

\item The CO\textsubscript{2} emission constraints implemented are representative of \gls{I2CNER} goals of an 80\% reduction in emissions from 1990 levels.
\end{enumerate}

\subsection{Model Data}
\begin{itemize}

\item \textbf{Emission coefficients} \cite{noauthor_electricity_2017}: The following data is in gCO$_2$/kWh (i.e. nuclear, solar and wind emissions from construction are averaged over the lifetime of the power-production process):

\begin{tabular}{|c|c|}
\hline
\textbf{Electricity source} & \textbf{Emissions coefficient}\\
\hline
Coal & 943 \\
\hline
Natural Gas & 599 \\
\hline
Oil & 738 \\
\hline
Solar & 38 \\
\hline
Wind & 25 \\
\hline
Nuclear & 21 \\
\hline
Geothermal & 13 \\
\hline
Hydropower & 11 \\
\hline
\end{tabular}

\item \textbf{\gls{LCoE}}\cite{noauthor_lazards_2017} : \gls{LCoE} data is appropriate for use in processes where a fixed amount of capacity has already been deployed i.e. the initial capacity. The following \gls{LCoE} data (in million USD/GWh) was used :\\

\begin{tabular}{|c|c|}
\hline
\textbf{Electricity source} & \textbf{LCoE}\\
\hline
Coal & 0.06 \\
\hline
Natural Gas & 0.08 \\
\hline
Oil & 0.39 \\
\hline
Solar & 0.161 \\
\hline
Wind & 0.144 \\
\hline
Nuclear & 0.1 \\
\hline
Geothermal & 0(fixed capacity) \\
\hline
Hydropower & 0 (fixed capacity) \\
\hline
\end{tabular}

\item \textbf{Detailed costs} \cite{noauthor_eia_2019} : While the models with conventional energy sources have part of the following cost structure, these somewhat inaccurate and highly idealized figures need to be revised based on the data in the \gls{ARFC} \gls{I2CNER} repository, especially for offshore and onshore wind(current data is for the US from the \gls{EIA} \cite{noauthor_eia_2019}), and for nuclear (to take construction delays into account).

\begin{tabular}{|c|c|c|c|}
\hline
\textbf{Electricity} & \textbf{Investment} & \textbf{Fixed \gls{OM}} & \textbf{Variable \gls{OM}}\\
\textbf{source} & \textbf{Cost (MUSD/GWh)} & \textbf{Cost (MUSD/GW)} & \textbf{Cost (MUSD/GWh)}\\
\hline
Coal & 3784 & 51.39 & 0.0072\\
\hline
Combined Cycle & 794 & 10.3 & 0.0021\\
\hline
Solar & 1783 & 22.46 & 0 \\
\hline
Onshore Wind & 2773 & 40.85 & 0\\
\hline
Offshore Wind & 8380 & 80.14 & 0\\
\hline
Nuclear & 1600 & 0.0165 & 0.00933\\
\hline
\end{tabular}

\item \textbf{Miscellaneous VEDA Parameters} \cite{kato_energy_2016} \cite{gargiulo_documentation_2005}:\\

\begin{tabular}{|c|c|c|c|c|}
\hline
\textbf{Electricity source} & \textbf{Efficiency} & \textbf{Utilization Factor} & \textbf{Lifetime (y)} & \textbf{\gls{PEAK} factor}\\
\hline
Coal & 0.6 & 0.95 & 60 & 1.0 \\
\hline
Combined Cycle & 0.5 & 0.95 & 60 & 1.0 \\
\hline
Solar & 0.20-0.27 & 0.13 & 20-25 & 0.42 \\
\hline
Onshore Wind & 0.9 & 0.23-0.25 & 25 & 0.20 \\
\hline
Offshore Wind & 0.9 & 0.31-0.32 & 25 & 0.20 \\
\hline
Nuclear & 0.9 & 0.95 & 60 & 1.0 \\
\hline
\end{tabular}

\end{itemize}

\section{Results}

Based on these assumptions, the model yields the following results for the years 2011-16, which are very close to the actual electricity generation figures supplied by \gls{EDMC}. \gls{LCoE}-based results for the entire time-period are present in the poster presented at the \gls{I2CNER} symposium \cite{chaube_dynamic_2019}.

\begin{figure}[H]
\centering
\includegraphics[scale=0.6]{elc-2016}
\caption{Electricity generated from different sources}
\label{elcic}
\end{figure}

The emitted CO\textsubscript{2} values for the period 2011-16 are as follows. The error is at most 5.7\% \ref{co2err}, which is due to the aforementioned absence of accurate data for 2011, 2012 and 2016. 

\begin{figure}[H]
\centering
\includegraphics[scale=0.6]{co2-2016}
\caption{CO2 emissions from electricity generation compared with actual emissions reported by \gls{MoE}, Japan}
\label{co2ic}
\end{figure}

\begin{figure}[H] \label{co2err}
\centering
\includegraphics[scale=0.6]{co2-err}
\caption{Relative error in CO2 emissions.}
\label{co2err}
\end{figure}

\begin{figure}[H]
\centering
\includegraphics[scale=0.6]{elc-2016}
\caption{Electricity generated from different sources}
\label{elcic}
\end{figure}

\begin{figure}[H]
\centering
\includegraphics[scale=0.6]{flat-bug}
\caption{Erroneous results obtained with discrete investment sizes in \gls{DSCINV} files.}
\label{flatbug}
\end{figure}

\begin{figure}[H]
\centering
\includegraphics[scale=0.6]{flat-good}
\caption{Accurate results obtained with semi-discrete investment sizes in \gls{DSCINV} files.}
\label{flatgood}
\end{figure}

\subsection{Post Hoc Analysis and Challenges} 

The project was significantly delayed due to the quality of the documentation and customer support provided by the VEDA developers. The primitive, black-box like nature of the software inhibits efficient debugging. Therefore, while data acquisition and organization proceeded at the originally suggested pace, the incorporation of this data into the model has been behind schedule. \\

The \gls{EDMC} is constantly being revised and updated, for both earlier (2010-2013) and later years(2016-). Hence, this data is not always consistent or complete, and often secondary sources must be used for verification.
\section{Future Work} 

\subsection{Critical Goals}
These targets are urgent and necessary.

\begin{enumerate}

\item \textbf{Replace \gls{LCoE} in all models with a detailed cost structure:} All models must incorporate investment and \gls{OM} costs.

\item \textbf{Remove pure oil based electricity generation from all models:} We do not think Japan will ever increase its dependence on oil due to its cost, emissions intensity and due to the Japanese goal of increasing energy independence. Current trends support this assumption. While a few models already exclude oil-based energy, other models should also replace it with combined-cycle electricity generation for the sake of consistency.

\item \textbf{Incorporate semi-discrete investment sizes:} As stated previously, all \gls{DSCINV} files must include semi-discrete capacity installation sizes for consistency.

\item \textbf{Restrict maximum wind capacity:} To align the model more closely with JWPA predictions \cite{heger_wind_2016}, the maximum allowed capacities for wind should be reduced in the respective \gls{MaxCAP} files.

\item \textbf{Associate wind (and solar) with natural gas/storage:} There may be two ways to accomplish this:

\begin{itemize}

\item \textbf{Define load curves for solar and wind:} This is the approach suggested by VEDA developers on their forum. However, the details of implementation for this particular approach are lacking in the \gls{TIMES} documentation. A successful implementation should incorporate the daily and seasonal variation of these electricity sources, and force the model to deploy natural gas or electricity storage to supplement wind and solar.

\item \textbf{Replace annually averaged capacity factors and/or \gls{PEAK} factors with seasonal (summer/winter) and diurnal (day/night) (i.e. SN,SD, WN, WD \cite{gargiulo_documentation_2005} ) capacity factors} : The model may then automatically deploy natural gas to supplement wind and solar. If seasonal versions of these factors exist, this would be the easiest solution. \gls{TIMES} Documentation Part II \cite{loulou_documentation_2005} may offer some insights in this regard.

\item \textbf{Define a direct relationship between the capacities of wind and natural gas}:It might be possible to define a direct equation between the capacity of renewables and natural gas. Since no straightforward way to do this is described in the VEDA documentation \cite{gargiulo_documentation_2005}, this would require utilization of the \gls{TIMES} documentation \cite{loulou_documentation_2005}, the VEDA attributes table, and quite possibly the assistance of the VEDA forum.

\end{itemize}

\item \textbf{Incorporation of Japan-specific costs for wind:} When incorporating \gls{JWPA} predictions, it will be necessary to split off-shore wind into fixed and floating types. The \href{https://github.com/arfc/i2cner/tree/master/data/japan_costs}{cost data} for this already exists in our repository thanks to Akari Minami, an undergraduate from Kyushu University who assisted with data collection and simulation during March 2019. This data needs to be sifted through and incorporated into our model.

\item \textbf{Revise \gls{CCS} costs:} Simplified \gls{CCS} electricity generation processes exist in our \gls{I2CNER} models. These emit only 10\% of the CO$_2$ that their corresponding fossil fuel technology emits \cite{kato_energy_2016}. In the model, these look like any other fossil fuel electricity generation process, except they are more expensive and have a significantly smaller emissions coefficient. Such an implementation does \textbf{not} include retrofitting of \gls{CCS} (i.e. adding point carbon capture capabilities to previously deployed fossil fuel capacity - discussed below). While we do not have Japan-specific data for \gls{CCS}, we can use cost data from the US and do one of the following:
\begin{itemize}

\item Neglect the difference between Japan and the US, assuming that the government will foot the bill of setting up the \gls{CCS} shipping network and offshore storage sites.

\item Roughly increase the cost by a small percentage, since setting up the Japanese \gls{CCS} shipping network and offshore storage sites will result in an increased cost per unit CO$_2$ captured and stored.

\item Attempt to conduct a rough ab-initio analysis to find the cost of capturing, transporting and storing one ton of CO$_2$. The cost of capturing CO$_2$ is more or less uniform and readily available \cite{kato_energy_2016}. The cost of transport and storage can be estimated by finding the cost of offshore-drilling to the depths necessary for CO$_2$ storage, the cost of pressurizing 1 ton of CO$_2$, and the cost of transporting a ton of cargo to offshore storage sites by ship. The exact costs for this vary based on the scale of the operation.

\end{itemize}

\item \textbf{Revise nuclear costs:} Current models include the ideal cost of nuclear, but actual costs are often higher due to delays in construction. This is accurately reflected in data from \gls{EIA} \cite{noauthor_eia_2019}, which already exists in our \href{https://github.com/arfc/i2cner/tree/master/data/japan_costs/fossil-ccs-nuc.xlsx}{repository}. This needs to be incorporated into our models to reduce over-deployment of nuclear.

\end{enumerate}

\subsection{Desired Goals}
These targets are not urgent but are required for improving our results.

\begin{itemize}

\item Make electricity demand process more realistic:

\begin{itemize}
\item At the very least, demand should increase at +1.7 \% per year until 2030 as per \gls{METI} projections \cite{noauthor_electricity_2017}, and should plateau afterwards until 2100.

\item More accurate data for 2030-2050 should be sought to further improve upon this, if possible.

\end{itemize}

\item Cost Analysis - Some metric to compare the transition costs for each scenario should be calculated and presented with our results. For example, the \gls{LCoE} for each scenario for different years (say 2030,2050,2100) could be calculated, or the total cost of the transition (investment+generation) could be juxtaposed for each scenario.

\item Incorporate more \gls{I2CNER} technology, such as perovskite solar cells, fuel cells for storage etc.

\end{itemize}

\subsection{Stretch Goals}
At this time, these targets are neither urgent nor necessary.

\begin{itemize}

\item \textbf{Implement \gls{CCS} retrofitting:} The modelling process for this is somewhat complicated. One would have to track CO$_2$ emitted from different fossil fuels separately by creating \gls{TIMES}  CO$_2$ commodities for coal, oil and petroleum, to ensure that CO$_2$ from non-fossil fuel sources is not captured by the model. Next, the process that converts this CO$_2$ to captured CO$_2$ and atmospheric CO$_2$ would need to be defined. The total amount of CO$_2$ captured may not be greater than the total capacity of the \gls{CCS} reservoirs around Japan \cite{kato_energy_2016}.\\
The data for retrofitting in Japan is not easily available. Generic \gls{CCS} data from other countries may be used if necessary.\\

\item \textbf{Sensitivity analysis:} To identify optimum thresholds for costs or parameters (like efficiency) of novel technologies, especially \gls{I2CNER} technology, to maximize their efficacy and penetration.

\end{itemize}


\bibliographystyle{ieeetr}
%\addbibresource{2018-chaube-i2cner-report-sept}
\bibliography{2018-chaube-i2cner-report-sept}

%\bibliographystyle{plain}

%\printbibliography

\end{document}
