%%%%%%%%%%%%%%%%%%%%%%%%%%%%%%%%%%%%%%%%%
% Jacobs Landscape Poster
% LaTeX Template
% Version 1.1 (14/06/14)
%
% Created by:
% Computational Physics and Biophysics Group, Jacobs University
% https://teamwork.jacobs-university.de:8443/confluence/display/CoPandBiG/LaTeX+Poster
% 
% Further modified by:
% Nathaniel Johnston (nathaniel@njohnston.ca)
%
% This template has been downloaded from:
% http://www.LaTeXTemplates.com
%
% License:
% CC BY-NC-SA 3.0 (http://creativecommons.org/licenses/by-nc-sa/3.0/)
%
%%%%%%%%%%%%%%%%%%%%%%%%%%%%%%%%%%%%%%%%%

%----------------------------------------------------------------------------------------
%	PACKAGES AND OTHER DOCUMENT CONFIGURATIONS
%----------------------------------------------------------------------------------------

\documentclass[final]{beamer}

\usepackage[scale=1.0]{beamerposter} % Use the beamerposter package for laying out the poster
\usetheme{confposter} % Use the confposter theme supplied with this template

\setbeamercolor{block title}{fg=dblue!80,bg=white} % Colors of the block titles
\setbeamercolor{block body}{fg=black,bg=white} % Colors of the body of blocks
\setbeamercolor{block alerted title}{fg=white,bg=dblue!70} % Colors of the highlighted block titles
\setbeamercolor{block alerted body}{fg=black,bg=dblue!10} % Colors of the body of highlighted blocks
% Many more colors are available for use in beamerthemeconfposter.sty

%-----------------------------------------------------------
% Define the column widths and overall poster size
% To set effective sepwid, onecolwid and twocolwid values, first choose how many columns you want and how much separation you want between columns
% In this template, the separation width chosen is 0.024 of the paper width and a 4-column layout
% onecolwid should therefore be (1-(# of columns+1)*sepwid)/# of columns e.g. (1-(4+1)*0.024)/4 = 0.22
% onecolwid should therefore be (1-(# of columns+1)*sepwid)/# of columns e.g. 
% (1-(3+1)*0.025)/3 = 0.3
% Set twocolwid to be (2*onecolwid)+sepwid = 0.464
% Set threecolwid to be (3*onecolwid)+2*sepwid = 0.708
\newcommand{\Cyclus}{\textsc{Cyclus}\xspace}%

\newlength{\sepwid}
\newlength{\onecolwid}
\newlength{\twocolwid}
\newlength{\threecolwid}
\setlength{\paperwidth}{36in} % A0 width: 46.8in
\setlength{\paperheight}{48in} % A0 height: 33.1in
\setlength{\textwidth}{34in} % A0 width: 46.8in
\setlength{\textheight}{46in} % A0 height: 33.1in
\setlength{\sepwid}{0.025\paperwidth} % Separation width (white space) between columns
\setlength{\onecolwid}{0.3\paperwidth} % Width of one column
\setlength{\twocolwid}{0.625\paperwidth} % Width of two columns
\setlength{\threecolwid}{0.95\paperwidth} % Width of three columns
\setlength{\topmargin}{-0.5in} % Reduce the top margin size
%-----------------------------------------------------------

\usepackage{graphicx}  % Required for including images

\usepackage{tabularx}
\newcolumntype{b}{X}
\newcolumntype{s}{>{\hsize=.5\hsize}X}
\newcolumntype{m}{>{\hsize=.75\hsize}X}
\newcolumntype{z}{>{\hsize=.65\hsize}X}

\usepackage{booktabs} % Top and bottom rules for tables
\usepackage{xspace}

\usepackage{tikz}
\usepackage{chronology}
\usetikzlibrary{arrows.meta}

\usetikzlibrary{positioning, arrows, decorations, shapes, calc }
% Define block styles
\tikzstyle{decision} = [diamond, draw, fill=blue!20, 
text width=4.5em, text badly centered, node distance=3cm, inner sep=0pt]

\tikzstyle{const} = [rectangle, draw, text centered, fill=orange!20]
\tikzstyle{data} = [rectangle, draw, text centered, fill=green!20]

\tikzstyle{block} = [rectangle, draw, text centered, fill=blue!20]
\tikzstyle{line} = [draw, -latex']
\tikzstyle{cloud} = [draw, ellipse,fill=red!20, node distance=6em,
minimum height=2em]



\usetikzlibrary{shapes.multipart}
\usetikzlibrary{positioning}


\setbeamertemplate{bibliography item}[text]

%----------------------------------------------------------------------------------------
%	TITLE SECTION 
%----------------------------------------------------------------------------------------

\title{Dynamic Transition Analysis with TIMES:\\ 
I\textsuperscript{2}CNER Initiative on Challenges in Energy Assessment and Energy Transitions} % Poster title

\author{Anshuman Chaube, James Stubbins, \textbf{Kathryn Huff}\\Energy Analysis Division}
\institute{University of Illinios at Urbana-Champaign, Department of Nuclear, Plasma, and Radiological Engineering, Urbana, IL 61801}
%----------------------------------------------------------------------------------------

\begin{document}

\addtobeamertemplate{block end}{}{\vspace*{2ex}} % White space under blocks
\addtobeamertemplate{block alerted end}{}{\vspace*{2ex}} % White space under highlighted (alert) blocks

\setlength{\belowcaptionskip}{2ex} % White space under figures
\setlength\belowdisplayshortskip{2ex} % White space under equations

\begin{frame}[t] % The whole poster is enclosed in one beamer frame

\begin{columns}[t,totalwidth=\threecolwid] % The whole poster consists of three major columns, the second of which is split into two columns twice - the [t] option aligns each column's content to the top


%=======================================================
% FIRST COLUMN BEGINS
%=========================================================


\begin{column}{\twocolwid} % The first two columns, two cols wide
\begin{columns}[t,totalwidth=\twocolwid] % Split up the first column into two, 
\begin{column}{\onecolwid} % The first column, one col wide

%----------------------------------------------------------------------------------------
%	OBJECTIVES
%----------------------------------------------------------------------------------------

\begin{alertblock}{Objectives}

Evaluate potential impact of novel energy technologies within Japan's energy system. Specifically:
\begin{itemize}
	\item Guide practical near term (2010-2050) transition strategies.
        \item Minimize carbon emissions within realistic constraints.
	\item Identify high impact technologies.
	\item Assess role of technology readiness.
	\item Predict impediments to strategically optimal technology deployment.
	\item Identify ideal timelines for energy system deployment, infrastructure development, high impact R\&D investment.
\end{itemize}

\end{alertblock}
%
%\begin{block}{Collaborators}
%
%I\textsuperscript{2}CNER collaborators will include primarily the members of 
%the Energy Analysis Division (EAD), including those members at other 
%institutes, universities or industries with connections to the EAD.   
%\end{block}

%----------------------------------------------------------------------------------------
%	METHODOLOGY
%----------------------------------------------------------------------------------------

\begin{block}{Introduction}
Previous work has compared the impact of innovative energy technologies in 
various world regions using \textbf{static} scenario analyses 
\cite{chyong_chi_dynamics_2009,feng_system_2013,kikuchi_simulation-based_2017,li_energy_2010,pambudi_impact_2017,pambudi_future_2016}.  
We will simulate \textbf{dynamic} transition scenarios 
\cite{jebaraj_review_2006,pfenninger_energy_2014} aimed at minimizing carbon 
emissions in Japan by 2050. These scenarios will include realistic constraints 
regarding technology readiness (in terms of generation, transmission \& 
storage) and will combine multiple technologies in a single heterogeneous 
system model.
\end{block}

\begin{block}{Methodology}
        \textbf{The Integrated MARKAL-EFOM System (TIMES)} model generator 
        \cite{loulou_documentation_2005} 
        \cite{seebregts_energy/environmental_2002} optimizes energy systems 
        using linear and mixed-linear algorithms. A user-defined objective 
        function (such as minimizing carbon emissions or costs) is solved within user defined constraints such as energy generation demand.  
\end{block}


%----------------------------------------------------------------------------------------

\end{column} % End of the first column

%===========================================================================================
% SECOND COLUMN BEGINS
%=============================================================================================

%----------------------------------------------------------------------------------------

\begin{column}{\onecolwid} % The second column


%----------------------------------------------------------------------------------------
%	Anticipated Results
%----------------------------------------------------------------------------------------

\begin{block}{Anticipated Results}

 \begin{itemize}
   \item Analysis results can be filtered by sector (commercial, industrial, residential, building etc) or by region.
   
   \item Many metrics are automatically postprocessed- such as energy intensity, thermal energy efficiency, transmission capacity. 
   
   \item Technology deployment transitions driven by constrained optimization will have valuable strategic value.    
 
 
 \end{itemize}  
 
 	
\end{block}





%----------------------------------------------------------------------------------------
%	RESULTS
%----------------------------------------------------------------------------------------


        \begin{alertblock}{Impact}
        \large Results will:
	\begin{itemize}
        \item {Optimize realistic decarbonization roadmaps.}
        \item {Identify potential transition bottlenecks.}
        \item {Help Japan's policymakers create timelines for R\&D investment and infrastructure development.}
        \item {Quantify system sensitivity to technology readiness.}
	\end{itemize}
        \end{alertblock}
   

%----------------------------------------------------------------------------------------


\end{column} % End of column 2, top half
\end{columns}
%=======================================================
% Double Wide Figure Column 
%=========================================================

%=======================================================
% FIGURE BEGINS
%=========================================================

% tell TikZ how to stack them (back to front)
\newlength{\figwidth}
\setlength{\figwidth}{8cm}

\begin{figure}
    \centering
    \scalebox{1.0}{
                \begin{tikzpicture}[>={Latex[width=6mm,length=6mm]},
                                node distance=\figwidth,
                                on grid,
                                align=center,
                                auto]
        % Place nodes
        \node [block] (times) {\textbf{TIMES Model Generator}};
        \node [cloud, below=\figwidth of times] (mod) {\texttt{MODEL} \\ heterogeneous \\ multi-technology \\ model of Japan};
                \node [cloud, above left=1.5\figwidth and 2*\figwidth of times]
                (dat) {\texttt{DATA}\\regarding both\\i$^2$cner and\\ conventional\\technologies};
                \node [data, above=1.5\figwidth of dat] (dat1) {Storage\\Capacity};
                \node [data, above=0.5\figwidth of dat1] (dat2) {Thermal\\Efficiency};
                \node [data, above=0.5\figwidth of dat2] (dat3) {Capacity\\factor};
                \node [data, above=0.5\figwidth of dat3] (dat4) {Availability\\factor};
                \node [data, right=\figwidth of dat1] (dat5) {Thermal\\capacity};
                \node [data, above=0.5\figwidth of dat5] (dat6) {Fuel costs};
                \node [data, above=0.5\figwidth of dat6] (dat7) {Construction\\costs};
                \node [data, above=0.5\figwidth of dat7] (dat8) {Operation\\costs};
                \node [data, above=0.5\figwidth of dat8] (dat9) {Maintenance\\costs};
                \node [data, right=\figwidth of dat5] (dat10) {Technology\\readiness};
                \node [data, above=0.5\figwidth of dat10] (dat11) {Construction\\time};
                \node [data, above=0.5\figwidth of dat11] (dat12) {Carbon\\intensity};
                \node [data, above=0.5\figwidth of dat12] (dat13) {Fuel needs};
        \node [cloud, above=\figwidth of times] (of) {\texttt{OBJECTIVE}\\\texttt{FUNCTION}};
        \node [block, above left=\figwidth of of] (min) {Minimize\\carbon emissions \\ from all sources};
        \node [block, above right=\figwidth of of] (max) {Maximize\\energy market\\ diversity};
        \node [cloud, above right=1.5*\figwidth and 2*\figwidth of times] (const) {\texttt{CONSTRAINTS} \\ e.g.: Deployed \\ sources must\\ meet energy \\ demand};
        \node [const, above=\figwidth of const ] (dem) {electricity\\demand growth};
        \node [const, left=\figwidth of dem ] (init) {initial\\condition (2010)};
        \node [const, above=0.5*\figwidth of dem] (infra) {infrastructure\\availability};
        \node [const, left=\figwidth of infra] (sec) {consumption\\by sector};
        \node [const, above=0.5*\figwidth of infra] (reg) {regional\\transmission};
        
        \begin{scope}[on background layer]
        \draw[->, ultra thick] let \p1=($(times)-(dat)$) in (dat) -- +(0,\y1)-- +(times);
        \draw[->, ultra thick] (of) -- (times);
        %\draw[->, ultra thick] let \p3=($(times)-(const)+(2,3)$),\p4=($(times)+(9,9)$),\p5=($(const)-(3,3)$) in (\x3,\y4) -- +(0,\y2)-- +(times);
        \draw[->, ultra thick] let \p2=($(times)-(const)$) in (const)-- +(0,\y2) -- + (times);
                \draw[->, ultra thick] (times) -- (mod);
        \draw[->, ultra thick] (min) -- (of);
        \draw[->, ultra thick] (max) -- (of);
        \draw[->, ultra thick] (dem) -- (const);
        \draw[->, ultra thick] (dat1) -- (dat);
        \draw[->, ultra thick] (dat2) -- (dat);
        \draw[->, ultra thick] (dat3) -- (dat);
        \draw[->, ultra thick] (dat4) -- (dat);
        \draw[->, ultra thick] (dat5) -- (dat);
        \draw[->, ultra thick] (dat6) -- (dat);
        \draw[->, ultra thick] (dat7) -- (dat);
        \draw[->, ultra thick] (dat8) -- (dat);
        \draw[->, ultra thick] (dat9) -- (dat);
        \draw[->, ultra thick] (dat10) -- (dat);
        \draw[->, ultra thick] (dat11) -- (dat);
        \draw[->, ultra thick] (dat12) -- (dat);
        \draw[->, ultra thick] (dat13) -- (dat);
        \draw[->, ultra thick] (reg) -- (const);
        \draw[->, ultra thick] (infra) -- (const);
        \draw[->, ultra thick] (sec) -- (const);
        \draw[->, ultra thick] (init) -- (const);
                \path[->] (mod) edge [ultra thick,out=330,in=0,looseness=10] (mod) node[above right=0.1\figwidth and 0.8\figwidth] {simulate\\2010-2050};
                %\draw (mod.east) to [->, ultra thick,looseness=10] (mod.south) node[above right=0.1\figwidth and 0.1\figwidth] {simulate\\2010-2050};
        \end{scope}
 \end{tikzpicture}
    }
    \caption{Basic methodology for dynamic simulation of Japan's energy system.}
\end{figure}



\end{column} % End of wide column, total length

\begin{column}{\onecolwid} % The third column

%--------------------------------------------------------
% PROJECT TIMELINE
%---------------------------------------------------------
\begin{block}{Timeline}
        \newcommand\ytl[2]{\parbox[b]{0.3\textwidth}{\hfill{\color{orange!90}\bfseries\sffamily #1}~$\cdots\cdots$~}\makebox[0pt][c]{$\bullet$}\vrule\quad 
\parbox[c]{0.7\textwidth}{\vspace{7pt}\color{dblue!80}\raggedright\sffamily #2\\[7pt]}\\[-3pt]}
\begin{table}
\centering
\begin{minipage}[t]{\linewidth}
\color{gray}
\rule{\linewidth}{1pt}
\ytl{Jan.~2018}{Project start: \hfill Literature Review.}
\ytl{Feb.~2018}{Data collection: \hfill Japan's current grid.}
\ytl{Mar.~2018}{Data collection: \hfill Static projections.}
\ytl{May.~2018}{Data collection: \hfill Conventional technologies.}
\ytl{Jun.~2018}{Data collection: \hfill i$^2$cner generation technology.}
\ytl{Jul.~2018}{Data collection: \hfill i$^2$cner efficiency technology.}
\ytl{Aug.~2018}{Data collection: \hfill i$^2$cner storage technology.}
\ytl{Sep.~2018}{Scenario simulation: \hfill 2010-2050 conventional.}
\ytl{Oct.~2018}{Scenario simulation: \hfill 2010-2050 i$^2$cner driven.}
\ytl{Dec.~2018}{Scenario simulation: \hfill 2010-2070.}
%\ytl{~~~~~}{~~~~}
\rule{\linewidth}{1pt}%
\\%
\bigskip
\ytl{~~~~~2019}{Sensitivity analysis: \hfill Vary key parameters.}
\bigskip
\rule{\linewidth}{1pt}%
\end{minipage}%
\end{table}

\end{block}



%----------------------------------------------------------------------------------------
%	CONCLUSION
%----------------------------------------------------------------------------------------

\begin{block}{Summary}
\begin{itemize}

\item Dynamic simulation of Japan's energy system in TIMES model generator using a heterogeneous model and realistic constraints will help develop near-term decarbonization strategies.

\item Policymakers will benefit from identification of high impact technologies, and creation of R\&D investment and infrastructure development timelines.

\item Simulations will quantify system sensitivity to technology readiness, and also account for secondary scenarios where decarbonization is not the main priority.

\end{itemize}

\end{block}


\begin{block}{References}
        {\footnotesize\bibliographystyle{abbrv} 
        \bibliography{poster}}
\end{block}

%----------------------------------------------------------------------------------------

%----------------------------------------------------------------------------------------
%	CONTACT INFORMATION
%----------------------------------------------------------------------------------------

\setbeamercolor{block alerted title}{fg=black,bg=norange} % Change the alert block title colors
\setbeamercolor{block alerted body}{fg=black,bg=white} % Change the alert block body colors

\begin{alertblock}{Contact Information}
\begin{itemize}
	\item Web: \href{arfc.github.io}{arfc.github.io}
	\item Email: \href{mailto:kdhuff@illinois.edu}{kdhuff@illinois.edu}
\end{itemize}

\end{alertblock}

%----------------------------------------------------------------------------------------
%	ACKNOWLEDGEMENTS
%----------------------------------------------------------------------------------------

\setbeamercolor{block title}{fg=norange,bg=white} % Change the block title color

\begin{block}{Acknowledgements}

This research is being performed using funding received
from the International Institute for Carbon Neutral Energy Research (I$^2$CNER) 
Initiative on Challenges in Energy Assessment and Energy Transitions at  the  
University of Illinois under Director Petros Sofronis.

\vspace{10mm}
\begin{center}
\begin{tabular}{cccc}
\includegraphics[scale=0.5]{arfc_logo.png} & \includegraphics[scale=0.5]{wpi_logo.png} & \includegraphics[scale=0.1]{ku_logo.png} &\includegraphics[scale=0.7]{i2cner_logo.png}
%\includegraphics[width=\linewidth, height=0.1\textheight]{i2cner_logo.png}
\end{tabular}
\end{center}


\end{block}

\end{column} % End of the final column

\end{columns}
\end{frame} % End of the enclosing frame

\end{document}
